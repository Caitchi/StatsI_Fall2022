\documentclass[12pt,letterpaper]{article}
\usepackage{graphicx,textcomp}
\usepackage{natbib}
\usepackage{setspace}
\usepackage{fullpage}
\usepackage{color}
\usepackage[reqno]{amsmath}
\usepackage{amsthm}
\usepackage{fancyvrb}
\usepackage{amssymb,enumerate}
\usepackage[all]{xy}
\usepackage{endnotes}
\usepackage{lscape}
\newtheorem{com}{Comment}
\usepackage{float}
\usepackage{hyperref}
\newtheorem{lem} {Lemma}
\newtheorem{prop}{Proposition}
\newtheorem{thm}{Theorem}
\newtheorem{defn}{Definition}
\newtheorem{cor}{Corollary}
\newtheorem{obs}{Observation}
\usepackage[compact]{titlesec}
\usepackage{dcolumn}
\usepackage{tikz}
\usetikzlibrary{arrows}
\usepackage{multirow}
\usepackage{xcolor}
\newcolumntype{.}{D{.}{.}{-1}}
\newcolumntype{d}[1]{D{.}{.}{#1}}
\definecolor{light-gray}{gray}{0.65}
\usepackage{url}
\usepackage{listings}
\usepackage{color}

\definecolor{codegreen}{rgb}{0,0.6,0}
\definecolor{codegray}{rgb}{0.5,0.5,0.5}
\definecolor{codepurple}{rgb}{0.58,0,0.82}
\definecolor{backcolour}{rgb}{0.95,0.95,0.92}

\lstdefinestyle{mystyle}{
	backgroundcolor=\color{backcolour},   
	commentstyle=\color{codegreen},
	keywordstyle=\color{magenta},
	numberstyle=\tiny\color{codegray},
	stringstyle=\color{codepurple},
	basicstyle=\footnotesize,
	breakatwhitespace=false,         
	breaklines=true,                 
	captionpos=b,                    
	keepspaces=true,                 
	numbers=left,                    
	numbersep=5pt,                  
	showspaces=false,                
	showstringspaces=false,
	showtabs=false,                  
	tabsize=2
}
\lstset{style=mystyle}
\newcommand{\Sref}[1]{Section~\ref{#1}}
\newtheorem{hyp}{Hypothesis}


\title{Problem Set 4}
\date{Due: December 4, 2022}
\author{Applied Stats/Quant Methods 1}


\begin{document}
	\maketitle
	\vspace{.25cm}
\section*{Question 1: Economics}
\vspace{.25cm}
\noindent 	
In this question, use the \texttt{prestige} dataset in the \texttt{car} library. First, we  run the following commands:

\begin{verbatim}
install.packages(car)
library(car)
data(Prestige)
help(Prestige)
\end{verbatim} 


\noindent We would like to study whether individuals with higher levels of income have more prestigious jobs. Moreover, we would like to study whether professionals have more prestigious jobs than blue and white collar workers.


\begin{enumerate}
	
	\item [(a)]
	Create a new variable \texttt{professional} by recoding the variable \texttt{type} so that professionals are coded as $1$, and blue and white collar workers are coded as $0$ (Hint: \texttt{ifelse}).
	
	\vspace{.25cm}
	
	\noindent Create dummy variables:
	
	\lstinputlisting[language=R, firstline=54, lastline=54]{PS4_CaitlinCooney_16322496.R}
	\vspace{.1cm}
	
	\noindent Create data frame to use for regression:
	
	\lstinputlisting[language=R, firstline=58, lastline=58]{PS4_CaitlinCooney_16322496.R}
	\vspace{.1cm}
	
	\noindent View data frame:
	
	\lstinputlisting[language=R, firstline=61, lastline=61]{PS4_CaitlinCooney_16322496.R}
	\vspace{.1cm}
	
	% latex table generated in R 4.2.1 by xtable 1.8-4 package% Fri Dec  2 16:22:25 2022
	
% latex table generated in R 4.2.1 by xtable 1.8-4 package
% Fri Dec  2 16:24:03 2022

\begin{table}[ht!]
\centering
\begin{tabular}{rrrrrrlr}  \hline & education & income & women & prestige & census & type & professional \\   
	\hline gov.administrators & 13.11 & 12351 & 11.16 & 68.80 & 1113 & prof & 1.00 \\   general.managers & 12.26 & 25879 & 4.02 & 69.10 & 1130 & prof & 1.00 \\   accountants & 12.77 & 9271 & 15.70 & 63.40 & 1171 & prof & 1.00 \\   purchasing.officers & 11.42 & 8865 & 9.11 & 56.80 & 1175 & prof & 1.00 \\   chemists & 14.62 & 8403 & 11.68 & 73.50 & 2111 & prof & 1.00 \\   physicists & 15.64 & 11030 & 5.13 & 77.60 & 2113 & prof & 1.00 \\    \hline
\end{tabular}
\end{table}
	
	\vspace{5cm}
	
	\item [(b)]
	Run a linear model with \texttt{prestige} as an outcome and \texttt{income}, \texttt{professional}, and the interaction of the two as predictors (Note: this is a continuous $\times$ dummy interaction.)
	
		\lstinputlisting[language=R, firstline=67, lastline=70]{PS4_CaitlinCooney_16322496.R}
	\vspace{.1cm}
	
	% Table created by stargazer v.5.2.3 by Marek Hlavac, Social Policy Institute. E-mail: marek.hlavac at gmail.com% Date and time: Fri, Dec 02, 2022 - 16:32:01
	\begin{table}[!htbp] 
		\centering   \caption{Prestige and income-professional Regression}   \label{} \begin{tabular}{@{\extracolsep{5pt}}lc} \\[-1.8ex]\hline \hline \\[-1.8ex]  & \multicolumn{1}{c}{\textit{Dependent variable:}} \\ \cline{2-2} \\[-1.8ex] & prestige \\ \hline \\[-1.8ex]  income & 0.003$^{***}$ \\   & (0.0005) \\   & \\  professional & 37.781$^{***}$ \\   & (4.248) \\   & \\  income:professional & $-$0.002$^{***}$ \\   & (0.001) \\   & \\  Constant & 21.142$^{***}$ \\   & (2.804) \\   & \\ \hline \\[-1.8ex] Observations & 98 \\ R$^{2}$ & 0.787 \\ Adjusted R$^{2}$ & 0.780 \\ Residual Std. Error & 8.012 (df = 94) \\ F Statistic & 115.878$^{***}$ (df = 3; 94) \\ \hline \hline \\[-1.8ex] \textit{Note:}  & \multicolumn{1}{r}{$^{*}$p$<$0.1; $^{**}$p$<$0.05; $^{***}$p$<$0.01} \\ \end{tabular} 
	\end{table} 
	
	\newpage
	\item [(c)]
	Write the prediction equation based on the result.
	
\noindent prestige = 21.142 + (0.003*income) + (37.781*professional) + (-0.002*37.7812800*0.0031709)
	
	\item [(d)]
	Interpret the coefficient for \texttt{income}.
	
	\noindent The coefficient for income is 0.003, which is positive. This indicates that as 
	the value of income increases, the mean of the dependant variable also tends to increase,
	and a one-unit shift in income (holding all other varables constant) causes a 0.003 unit 
	increase in prestige. 
	
	\vspace{.25cm}	
	\item [(e)]
	Interpret the coefficient for \texttt{professional}.
	
	\noindent The coefficient for professional is 37.781, which is positive. This indicates that
	income is higher for the dummy variable 'professional' than for the reference group 
	(white and blue collar workers), and that type 'professional' indicates a 37.781 
	increase in income.
	
	\item [(f)]
	What is the effect of a \$1,000 increase in income on prestige score for professional occupations? In other words, we are interested in the marginal effect of income when the variable \texttt{professional} takes the value of $1$. Calculate the change in $\hat{y}$ associated with a \$1,000 increase in income based on your answer for (c).
	
		\lstinputlisting[language=R, firstline=98, lastline=99]{PS4_CaitlinCooney_16322496.R}
	\vspace{.1cm}
	
	\begin{verbatim}
		[1] 61.92276
	\end{verbatim}

\noindent The change in prestige associated with a \$1,000 increase in income is 61.92276.
	
	\vspace{.25cm}
	
	
	\item [(g)]
	What is the effect of changing one's occupations from non-professional to professional when her income is \$6,000? We are interested in the marginal effect of professional jobs when the variable \texttt{income} takes the value of $6,000$. Calculate the change in $\hat{y}$ based on your answer for (c).
	
		\lstinputlisting[language=R, firstline=106, lastline=113]{PS4_CaitlinCooney_16322496.R}
	\vspace{.1cm}
	
	\begin{verbatim}
		[1] 37.781
	\end{verbatim}

\noindent This means that changing one's occupations from non-professional to professional when income is \$6,000 leads to a 37.781 increase in prestige.
	
\end{enumerate}

\newpage

\section*{Question 2: Political Science}
\vspace{.25cm}
\noindent 	Researchers are interested in learning the effect of all of those yard signs on voting preferences.\footnote{Donald P. Green, Jonathan	S. Krasno, Alexander Coppock, Benjamin D. Farrer,	Brandon Lenoir, Joshua N. Zingher. 2016. ``The effects of lawn signs on vote outcomes: Results from four randomized field experiments.'' Electoral Studies 41: 143-150. } Working with a campaign in Fairfax County, Virginia, 131 precincts were randomly divided into a treatment and control group. In 30 precincts, signs were posted around the precinct that read, ``For Sale: Terry McAuliffe. Don't Sellout Virgina on November 5.'' \\

Below is the result of a regression with two variables and a constant.  The dependent variable is the proportion of the vote that went to McAuliff's opponent Ken Cuccinelli. The first variable indicates whether a precinct was randomly assigned to have the sign against McAuliffe posted. The second variable indicates
a precinct that was adjacent to a precinct in the treatment group (since people in those precincts might be exposed to the signs).  \\

\vspace{.5cm}
\begin{table}[!htbp]
	\centering 
	\textbf{Impact of lawn signs on vote share}\\
	\begin{tabular}{@{\extracolsep{5pt}}lccc} 
		\\[-1.8ex] 
		\hline \\[-1.8ex]
		Precinct assigned lawn signs  (n=30)  & 0.042\\
		& (0.016) \\
		Precinct adjacent to lawn signs (n=76) & 0.042 \\
		&  (0.013) \\
		Constant  & 0.302\\
		& (0.011)
		\\
		\hline \\
	\end{tabular}\\
	\footnotesize{\textit{Notes:} $R^2$=0.094, N=131}
\end{table}

\vspace{.5cm}
\begin{enumerate}
	\item [(a)] Use the results from a linear regression to determine whether having these yard signs in a precinct affects vote share (e.g., conduct a hypothesis test with $\alpha = .05$).
	
	H0: B1 (the slope of the regression line for the effect of living in a precinct 
	assigned lawn signs on proportion of vote to Ken Cuccinelli) = 0
	
	H1: B1 (the slope of the regression line for the effect of living in a precinct 
	assigned lawn signs on proportion of vote to Ken Cuccinelli) != 0
	
	$\alpha = .05$

	\newpage
	\noindent Get the test statistic:
	
	\lstinputlisting[language=R, firstline=136, lastline=136]{PS4_CaitlinCooney_16322496.R}
	\vspace{.1cm}
	
	\noindent Get critical value at 0.05:
	
	\lstinputlisting[language=R, firstline=140, lastline=146]{PS4_CaitlinCooney_16322496.R}
	\vspace{.1cm}
	
	\noindent Since t1 of 2.625 is greater than p value of 0.01, we fail to reject the null 
	hypothesis that the slope of the regression line for the effect of living in a precinct 
	assigned lawn signs on proportion of vote to Ken Cuccinelli  = 0 at a 0.05\% significance level. 
	
	\item [(b)]  Use the results to determine whether being
	next to precincts with these yard signs affects vote
	share (e.g., conduct a hypothesis test with $\alpha = .05$).
	
	H0: B2 (slope of the regression line for the effect of living in a precinct 
	adjacent to lawn signs on proportion of vote to Ken Cuccinelli) = 0
	H1: B2 (the slope of the regression line for the effect of living in a precinct 
	adjacent to lawn signs on proportion of vote to Ken Cuccinelli) != 0
	
		$\alpha = .05$
	
	Get the test statistic 
	
	\lstinputlisting[language=R, firstline=164, lastline=165]{PS4_CaitlinCooney_16322496.R}
\vspace{.1cm}
	
	Get critical value at 0.05
	
		\lstinputlisting[language=R, firstline=168, lastline=174]{PS4_CaitlinCooney_16322496.R}
	\vspace{.1cm}
	
	\noindent Since t2 of 3.231 is greater than p value of 0.002, we fail to reject the null 
	hypothesis that the slope of the regression line for the effect of living in a precinct 
	adjacent to lawn signs on proportion of vote to Ken Cuccinelli = 0 at a 0.05\% significance level. 
		\vspace{.25cm}
	
	\item [(c)] Interpret the coefficient for the constant term substantively.
	\vspace{.25cm}
	
	\noindent In this model, a constant of 0.302 indicates that the value that would be 
	predicted for the proportion of the vote to go to Ken Cucinelli if all the 
	independent variables (yard signs and yard sign adjacent) were simultaneously 
	equal to zero is 0.302. 
	
		\vspace{.25cm}
	\item [(d)] Evaluate the model fit for this regression.  What does this	tell us about the importance of yard signs versus other factors that are not modeled?
	
	The R$^2$ for this model is 0.094. This low R$^2$ value tells us that compared to other factors not 
	modeled in this regression, yard signs have little impact on proportion of vote going to the opponent. 
	
	
\end{enumerate}  


\end{document}
